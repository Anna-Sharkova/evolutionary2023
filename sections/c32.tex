\section{Теоремы класса “No Free Lunch”.}
Не существует решения, которое позволит нам выиграть и во времени и по памяти!


Если рассмотреть:
\begin{itemize}
    \item все функции из конечного пространства X в конечное пространство Y
    \item все алгоритмы, генерирующие запросы к функции на основе результатов предыдущих запросов, и \textbf{не делающих никакой запрос более одного раза}
\end{itemize}

То
\begin{itemize}
    \item ни один алгорим не может быть строго лучше никакого другого
    \item \textbf{средняя производительность алгоритмов строго одинакова}
\end{itemize}

Как понимать теоремы No Free Lunch
\begin{itemize}
    \item Алгоритмы надо создавать под конкретные классы задач (явно или неявно) 
    \item «Все функции» — очень сложный объект
    \begin{itemize}
        \item Про большую часть функций никогда не будет ставится задача их оптимизации, потому что описание этих функций не помещается в известную Вселенную
        \item Строгие требования к классам функций, для которых такие теоремы верны, нарушаются почти для любого реалистичного класса функций
    \end{itemize}
\end{itemize}


Как не надо понимать теоремы «No Free Lunch»
\begin{itemize}
    \item «Не существует наилучшего алгоритма, поэтому давайте разработаем новый
    \begin{itemize}
        \item Неочевидно, что для алгоритма получится найти реалистичный класс задач, который он будет решать хорошо
    \end{itemize}
    \item «Эволюционные алгоритмы бесполезны — они не лучше случайного поиска»
    \begin{itemize}
        \item Неверно: встречающиеся на практике классы задач существенно проще, чем те, которые необходимы для выполнения теорем
        \item Возможное исключение: некоторые задачи, связанные с криптографией    
    \end{itemize}
\end{itemize}
