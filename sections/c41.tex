\section{Меметические алгоритмы. Гиперэвристики.}
\subsection{Меметические алгоритмы.}
Существует такой подвид семейства генетических алгоритмов,как \textbf{Меметические}. Максимально неудачное название разновидности эволюционного алгоритма. Такого вида алгоритмы скорее избавляются от присутствия генератора сучайных чисел. Часто применяется в том случае, когда для текущей задачи невозможно подобрать оптимальный генератор.
\begin{itemize}
    \item Вместо оператора мутации используется локальный поиск или жадный алгоритм или их совокупность.
    \begin{itemize}
        \item Начало работы: особь, подвергающаяся оператору. 
        \item Нередко устанавливается ограничение на время работы
    \end{itemize}
    \item Особенность №1: мутация больше не может поддерживать разнообразие
    
        Т.к. нет мутации популяция больше не может поддерживать свое собственное разнообразие за их счет. Почему? Многие операторы мутации поддерживают многократное применение. У нас итеративный процесс, поэтому разнообразие достигается за счет многократного повторения совокупности операторов мутации и последующего скрещивания. Так вот, в меметических алгоритмах из-за замены оператора мутации на другие алгоритмы теряется разнообразие. Поэтому используются более явные методы подержания разнообразия до генерации популяции. На каждой итерации отслеживаем величину функции применимости и по выбранной стратегии выкидываем из популяции те особи, которые не пригодня для дальнейшего поддержания разнооразия. А на их месте появляются особи, которые получены через догенерацию. (это пример работы генетического алгоритма)
        \begin{itemize}
        \item Двукратное применение оператора может быть эквивалентно однократному
        \item Рационально использовать явные методы поддержания разнообразия
        \item Скрещивание: Главное - попасть в зону притяжения хорошего локального оптимума.(Иногда - можно считать, что аргументы локальные оптимумы) Т.е. при скрещивания тех особей, порождение которых не было затронуто мутаиционными механизмами, необходимо отслеживать выход каждой из особей, полученных после скрещивания за определенную область. Соответственно, если получившаяся особь имеет величину функции применимости выше, чем у полученных особей до этого, то значит мы можем спуститься в эту область и далее искать в ней особь, которая будет оптимальна в данном локальном участке.
    \end{itemize}
    \item Особенность №2: мутация может делать дополнительные вычисления приспособленности. 
    
    Т.е. в силу того, что в операторе мутации теперь отсутствует генератор псевдослучайных чисел, мы можем можем одновременно с применением мутации вычислять функцию применимости. Это своего рода корректировка траетории вычисления на ходу. Во время мутации есть определенные критерии, которые соизмеримы с функцией вычислимости и если же критерий выдает осмысленнозначимые величины, то мутация протекает эффективно, если нет - то на этапе мутции можно эту осоь откидывать.
    \begin{itemize}
    \item Нужна аккуратность в сравнении с другими алгоритмами!
    \end{itemize}
\end{itemize}
\subsection{Гиперэвристики.}
Если в той области, в том пространстве, где необходимо решить задачу и эта задача целеноправленно подходит под генетические алгоритмы, однако никак не удается найти тот генератор, который удет адекватно вести в течении всео итеративного процесса, то рекомендуется обратиться к семействам  меметических алгоритмов.\\
Помимо комбинаций генетических алгоритмов существуют также \textbf{гиперэвристики}. Это что-то вроде прогнозирования итеративного процесса в течение всей его протяженности. Т.е. заранее производится подбор результатоа, которые псевдогарантируются по исполнению этого итеративного процесса. Благодаря такому прогнозированию, мы можем оценивать эффективность каждого алгоритма на начальных этапах.\\
Методы принятия решений для работы с несколькими алгоритмами
\begin{itemize}
    \item \textbf{Селективные гиперэвристики}
    \begin{itemize}
        \item Решают, какой алгоритм из имеющихся запустить, в зависимости от особенностей задачи. Имеют в себе много механизмов  представления: работают и в векторах, и в графах, и в битовых строках, и т.д.
        \item Можно интерпретировать как задачу машинного (мета)обучения
        \item Могут быть как оффлайновыми, так и онлайновыми (зависит от сложности прогноза)
    \end{itemize}
    \item \textbf{Конструктивные гиперэвристики}
    \begin{itemize}
        \item Вместо выбора алгоритма строят (подбирают) операторы мутации, скрещивания, селекции, а также структуру алгоритмов оптимизации в зависимости от того, какая задача решается, и что в процессе происходит.\\
        Благодаря такой гиперэвристике можем на ходу отсалеживать эффективность того или иного оператора и изменять его в течение самого итеративного процесса.
        \item Как правило, ограничиваются операторами того или иного вида, что упрощает структуру таких алгоритмов и их применение (т.к.чем больше операторов, тем дольше происходит гиперэвристика). Может быть ограничение на количество или введение уровня значимости (каждый оператор отождествляется с некоторой величиной, которая изображает эффективность и участие в получении особи с высоким значением функции применимости. Если же такого механизма не будет, то конструтивная гиперэвристика как правило вырождается, потому что происходит неявный подбор параметров, т.е. вырождается в простейший перебор.
    \end{itemize}
\end{itemize}      
