\section{Меметические алгоритмы. Гиперэвристики.}
\subsection{Меметические алгоритмы.}
Существует такой подвид семейства генетических алгоритмов,как \textbf{Меметические}. Максимально неудачное название разновидности эволюционного алгоритма. Такого вида алгоритмы скорее избавляются от присутствия генератора сучайных чисел. Часто применяется в том случае, когда для текущей задачи невозможно подобрать оптимальный генератор.
\begin{itemize}
    \item Вместо оператора мутации используется локальный поиск или жадный алгоритм или их совокупность.
    \begin{itemize}
        \item Начало работы: особь, подвергающаяся оператору. 
        \item Нередко устанавливается ограничение на время работы
    \end{itemize}
    \item Особенность №1: мутация больше не может поддерживать разнообразие
        Т.к. нет мутации популяция больше не может поддерживать свое собственное разнообразие за их счет. Почему?Многие операторы мутации поддерживают многократное применение. У нас итеративный процесс, поэтому разнообразие достигается за счет многократного повторения совокупности операторов мутации и последующего скрещивания. Так вот, в меметических алгоритмах из-за замены оператора мутации на другие алгоритмы теряется разнообразие. Поэтому используются более явные методы подержания разнообразия до генерации популяции.
        \begin{itemize}
        \item Двукратное применение оператора может быть эквивалентно однократному
        \item Рационально использовать явные методы поддержания разнообразия
        \item Скрещивание: Главное - попасть в зону притяжения хорошего локального оптимума.(Иногда - можно считать, что аргументы локальные оптимумы)
    \end{itemize}
    \item Особенность №2: мутация может делать дополнительные вычисления приспособленности
    \begin{itemize}
    \item Нужна аккуратность в сравнении с другими алгоритмами!
    \end{itemize}
\end{itemize}
\subsection{Гиперэвристики.}
Методы принятия решений для работы с несколькими алгоритмами
\begin{itemize}
    \item Селективные гиперэвристики
    \begin{itemize}
        \item Решают, какой алгоритм из имеющихся запустить, в зависимости от особенностей задачи
        \item Можно интерпретировать как задачу машинного (мета)обучения
        \item Могут быть как оффлайновыми, так и онлайновыми
    \end{itemize}
    \item Конструктивные гиперэвристики
    \begin{itemize}
        \item Строят операторы мутации, скрещивания, селекции, а также структуру алгоритмов оптимизации в зависимости от того, какая задача решается, и что в процессе происходит
        \item Как правило, ограничиваются операторами того или иного вида, что упрощает структуру таких алгоритмов и их применение
    \end{itemize}
\end{itemize}      
