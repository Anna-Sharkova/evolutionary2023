\section{Адаптация параметров в эволюционных стратегиях.}

 Можно выделить следующие параметры в эволюционных алгоритмах и их возможные методы адаптации:

\begin{center}
    \begin{tabular}{ |  l | p{5cm} |}
    \hline
    Параметры &  Возможные методы адаптации \\ \hline
    $\mu$ - количество родителей  &  Часто увеличивают при перезапуске 
    \\ \hline
    $\lambda$ - количество потомков &   Может увеличиваться или уменьшаться в процессе оптимизации \\ \hline
    Параметры мутации * &  1.	Сила мутации может быть записана в особи и также эволюционировать
    2.	Сила мутации может зависеть от траектории оптимизации, например, от изменений приспособленности. \\
    \hline
    \end{tabular}
\end{center}

* Параметры мутации, например “сила мутации" 
 может быть маленькая, средняя или большая.

Пример с силой мутации: 

\begin{itemize}
      \item Если более чем 1/5 потомков более приспособлены, чем их родители, значит ведется локальная оптимизация слишком сильна и надо увеличить силу мутации. 
      \item Если менее чем 1/5 потомков более приспособлены, чем их родители, то глобальная оптимизация слишком сильна и надо уменьшить силу мутации.
   \end{itemize}
