\section{Графовые представления и операторы над ними.}
Как мы можем представлять графы:
\begin{itemize}
	\item Матрица смежности
	\item Матрица смежности с реберной маской
	\item Списки/множества вершин и списки/множества ребер
\end{itemize}
В качесnве операций мутации можно использовать:
\begin{itemize}
	\item Добавить/удалить ребро, изменить метку на ребре
	\item Добавить/удалить вершину, изменить метку на вершине
	\item Повторить несколько раз (например, l ∼ B(n, 1/n) или P[l = k] ∼ k− β ), чтобы сделать глобальный оператор
\end{itemize}
Кроссоверы:
Если графы имеют разное множество вершит то можно использовать слияние: (V,E) ← Merge((V,E),(Vj,Ej))
Для этого мы сконкатенируем списки ребер и для каждой вершины v′∈V′ с некоторой вероятностью p выбираем случайную вершину из v∈V и склеиваем v и v′
Другим методом является Алгоритм Neat в котором каждое ребро хранит дату своей генерации и при кроссовере мы из ребер с одинаковой датой выбираем какой то одно. 
Непрямое (генеративное) кодирование:
дея состоит в том что особь порождаем граф. Это может быть:
\begin{itemize}
	\item Кодирование грамматикой. С помощью каких то правил особь порождает граф однако количество правил должно быть ограниченно 
	\item Lindenmayer’s L-Systems
	\item Много разных других
\end{itemize}
