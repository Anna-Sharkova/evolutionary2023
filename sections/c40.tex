\section{Комбинирование алгоритмов.}
Эволюционные (и не только) алгоритмы можно комбинировать.\\
Построение модели генетического алгоритма в композиции.\\
Примеры схожих подходов: в машином обучении это беггинги, бустинги. С точки зрения эволюционного программирования такой подход тоже имеет место быть.\\
В комбинациях заложено несколько признаков:
\begin{itemize}
    \item Более естественный взгляд на вещи: не считать, что тот или иной оператор должен обязательно использоваться в конкретном алгоритме.
    
    у нас есть оператор мутации, оператор скрещивания, оператор ображения к функции применимости, оператор генерации начальной популяции. Это стандартные составляющие генетического алгоритма. Комбинация может проводиться путем задания совокупности операторов, т.е. мы можем иметь несколько операторов мутации, несколько операторов скрещивания, а можем иметь несколько моделей в том смысле, что у одной модели могут быть операторы мутации, а у другой - нет, и тоже самое со скрещиванием. Так или иначе коминирование алгоритмов и соответствующих моделей направлено на максимизацию тех показателей, которых добивается та или иная модель по отдельности.(в том числе и во временном периоде)
    \item Один алгоритм может работать лучше в начале, другой в конце процесса оптимизации (Часто объясняется через дихотомию Exploration vs Exploitation)
    \item Можем интегрировать в эволюционные алгоритмы различные «жадные» стратегии, известные в предметной области. 
    \begin{itemize}
        \item Жадные стратегии эффективные локальные оптимизаторы
        \item Если интегрировать их с эволюционными алгоритмами, можно получать эффективные глобальные оптимизаторы
        \item в том числе можно комбинировать с классическим стеком алгоритмов
    \end{itemize}
\end{itemize}
Важно понимать, что при комбинации алгоритмов необходимо четко прослеживать цель той самой комбинации. Т.е. нельзя перебирать пары, составляющие  алгоритмов, пытаясь эмпирическим путем какое-то наилучшее решение. В начале мы делаем какие-то исследования, некий прогноз возможных комбинаций, некий тестовый прогон на небольшом количестве итераций и затем реализуем полностью комбинацию и прогоняем ее на всех этапах поиска глобального оптимума.
