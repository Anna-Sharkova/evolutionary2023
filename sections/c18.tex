\section{UMDA, PBIL, cGA, многомерные EDA.}

UMBA – Univariate Marginal Distribution Algorithm (Алгоритм одномерного предельного распределения).

Идея: генерируем выборку из $\lambda$ особей из распределения, выбираем $\mu$ лучших особей из $\lambda$ согласно функции приспособленности и задается критерий обновления модели: $$M_(_t_+_1_) =  1/\mu + \sum_{i=1}^{\mu} P_i $$

Особенности: 

\begin{itemize}
      \item Предыдущая модель выбрасывается целиком, 
      \item Склонен к стагнации, особенно при малых $\mu$. Чтобы это предотвратить вероятности в модели обрезаются по интервалу: 1/n; 1-1/n, где  n -  размер задачи.
   \end{itemize}

PBIL - Population Based Incremental Learning (Дополнительное обучение на основе популяции)

Идея: генерируем выборку из $\lambda$ особей из распределения, выбираем $\mu$ лучших особей из $\lambda$ согласно функции приспособленности и задается критерий обновления модели: $$M_(_t_+_1_) =  (1 - \gamma)*M_t +\gamma/\mu\sum_{i=1}^{\mu} P_i $$

Особенности:
\begin{itemize}
      \item Есть скорость обучения $\gamma$, которая определяет, насколько быстро модель обновляется.
      \item Если скорость обучения обратно пропорциональна количеству итераций, то в какой-то момент после кол-ва итераций знания будут делится на константный множитель.
      \item На конечный результат влияет вся последовательность моделей в отличие от прошлого алгоритма.
   \end{itemize}



cGA - Compact Genetic Algorithm (Компактный генетический алгоритм)

Идея: генерируем выборку из $\lambda=2$ особей из нашего распределения, будем считать u – лучшей особью, а v – худшей. Обновлять модель будем следующим образом:  $$M_(_t_+_1_) = M_t +\gamma*(u-v) $$ но необходимо перед использованием поправить вероятности в $M_(_t_+_1_)$

Особенности: 

\begin{itemize}
      \item Скорость обучения  $\gamma$ определяет, насколько быстро модель обновляется.
      \item Часто  $\gamma = 1/n$ 
   \end{itemize}

Многомерные EDA

Идея: Они используют более сложные модели, которые способны хранить и обновлять знания о зависимостях и ковариациях генов. 

Примеры:

\begin{itemize}
      \item Mutual Information Maximizing Input Clustering (MIMIC)
      \item Bivariate Marginal Distribution Algorithm (BMDA)
      \item Extended Compact Genetic Algorithm (ECGA)
   \end{itemize}

