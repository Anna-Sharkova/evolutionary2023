\section{Коэволюция.}
\textbf{Коэволюция}(в биологии) - эволюция двух и более видов таким образом, что они становятся:
\begin{itemize}
    \item соревновательная\\
    более и более эффективными друг против друга (хищник жертва, хозяин паразит)
    \item кооперативная\\
    более и более эффективными в совместном существовании (цветок опылитель)
\end{itemize}
Как правило и в том и в другом случае фенотипы обладают свойством парности. Обязательна привязка один к одному в парах и, как правило, эти пары не зависимы. Они могут быть в любом четном количестве, но так или иначе подразумевается, что их синергия или борьба независима друг от друга. т е есть отдельные парные конструкции которые образуют между собой либо синергию, либо конкуренцию, но, если мы говорим про внешнее взаимодействие, то они выступают как единая особь, т.е. грубо говоря мы взяли 10 особей - 5 пар и внутри пар проявляется коэволюция, но сами пары - как единые особи. \\
\\
\textbf{Коэволюция} (в оптимизации) - решение задачи оптимизации путем искусственной эволюции (конкурентнозначимой метрики) одного или более типов решений, таким образом, что приспособленность каждого решения зависит от других решений
\begin{itemize}
    \item соревновательная\\
    приспособленность определяется тем, насколько эффективно решение соревнуется с другими. Есть некоторая популяция из n особей и сами особи представляют собой "подособи" и в них работает соревновательная метрика. (самое простое - операции больше/меньше). Вытеснение особи из пары или из группы.
    \item кооперативная\\
    приспособленность определяется тем, насколько эффективно решение кооперирует с другими (самое простое - сложить значения функции применимости). Поиск максимальной синергии и вывод этой пары или четной группы из популяции как лучших особей - более подходящих и склонных к оптимому. 
\end{itemize}
\textbf{Основные типы коэволюции} \\
\textbf{Соревновательная коэволюция с 1 популяцией}
\begin{itemize}
    \item Все особи полученные на каждой итерации являются решениями одной и той же задачи. \\
    Т.е. не может быть чтобы, чтобы какая-либо особь отклонилась от заданной области и препятствовала бы нахожденю оптимума. Не может произойти вырождения или регрессии от оптимума к которому стремится один итеративный процесс.
    \item Все особи имеют один и тот же тип.\\
    (получены одинаковым механизмом генерации и проходят через одинаковые операторы мутации и скрещивания.)
    \item Приспособленность вычисляется путем прямого соревнования между особями.\\
    Т.е. функция применимасти работает для каждой особи, но также присутствует и внешний фактор отбора, который выдает лучшее значение функции применимости для особей.
\end{itemize}
\textbf{Соревновательная коэволюция с 2 популяциями}
\begin{itemize}
    \item Особи бывают двух различных типов\\
    Могут как подвергаться разным операторам мутации, кроссовера, так и принадлежать разным подпространствам. Например, один тип - битовые строки, а второй - линейных комбинаций.
    \item Особи различных типов решают различные задачи
    \item Приспособленность особей типа A вычисляется путем соревнования с особями типа A
\end{itemize}
\textbf{Кооперативная коэволюция (со многими популяциями)}

\begin{itemize}
    \item Есть N различных типов особей. Могут быть с различной историчностью (относительно новыми или неизменяемые с начала итеративного процесса)
    \item Решение исходной задачи: сборка из (минимум) одного решения каждого типа. Некая комбинация, которая представляет собой совокупность  тех особей, которые вместе дают лучший результат функции применимости.
    \item Приспособленность особей вычисляется по тому, как они кооперируют с особями других типов
\end{itemize}
\textbf{Приспособленность при коэволюции}\\
Описание возможной ситуации
\begin{itemize}
    \item Текущее решение заведомо лучше, чем 10 итераций назад. 
    \item При коэволюции кол-во итераций существенно меньше, чем при стандартных моделях генетических алгоритмах.
    \item Однако его приспособленность примерно такая же: враг не дремлет!
    \item Абсолютной приспособленности (значение выше которого нельзя подняться) нет как определять, когда остановиться? Есть некий критерий, который накладывается внешне на каждую 10ю/15ю итерцию и оценивает глобальность оптимума к которому пришли пришли конкурирующий/кооперирующие метрики.
\end{itemize}
Различные функции приспособленности
\begin{itemize}
    \item Внутренняя приспособленность в соревновании/кооперации. Можем отслеживать между различными типами особей. Нужен порог значимости при отборе комбинаций.
    \item Внешняя приспособленность «сама по себе». Иногда бывает непросто определить
\end{itemize}
Важное следствие
\begin{itemize}
    \item Приспособленность приходится пересчитывать заново на каждой итерации
\end{itemize}
