\section{Поддержание разнообразия популяции.}

\begin{itemize}
    \item \textbf{Пример использования:} обычная оптимизация, но нужно больше разнообразия (или постановка задачи мультимодальная)
    \item \textbf{Решение:} целенаправленно понижать приспособленность особей, слишком похожих друг на друга
    \begin{itemize}
        \item Похожи по приспособленности — наивный подход 
        \item Похожи по генотипам — вторая по простоте, не так наивно
        \item Похожи по фенотипам/поведению — в идеале надо делать так
    \end{itemize}
\end{itemize}

\textbf{Базовые подходы}
\begin{itemize}
    \item \textbf{«Fitness sharing»}
    \begin{itemize}
        \item Параметры: минимальное допустимое расстояние $\sigma$, функция расстояния $d(i, j)$
        \item«Sharing function»: $s(i,j) = 1 - \frac{d(i,j)}{\sigma}^\alpha$ если $d(i,j)< \sigma$, иначе 0
        \item Пересчет приспособленности: $F'(i) = \frac{\Sigma F(i)^\beta}{_js(i,j)}$
    \end{itemize}
    \item \textbf{«Crowding»}: приспособленность не меняется, меняется оператор отбора 
    \begin{itemize}
        \item \textbf{Restricted Tournament Selection}
        \begin{itemize}
            \item Добавляем особи в новую популяцию по одной
            \item Для каждой особи ищем турнирным отбором похожую особь
            \item Если найденная похожая особь хуже, заменяем еетекущей добавляемой особью
        \end{itemize}
        \item \textbf{Deterministic Crowding}
        \begin{itemize}
            \item Потомок заменяет наиболее похожего на себя родителя, если приспособленность потомка лучше    
        \end{itemize}
    \end{itemize}
\end{itemize}


\textbf{Специализированные методы}
\begin{itemize}
    \item \textbf{В дифференциальной эволюции}
    \begin{itemize}
        \item DE уже достаточно неплохо справляется с разделением на ниши
        \item Дополнительное улучшение: выбор ближайшего соседа (а не случайной особи) в качестве базового решения
    \end{itemize}
    \item \textbf{Кластеризация}: заимствование методов машинного обучения + модификация операторов отбора
    \item \textbf{Nearest Better}: эвристика для кластеризации, заточенная под эволюционные вычисления
    \begin{itemize}
        \item Для каждой особи, кроме наилучшей, находим ближайшую особь, приспособленность которой лучше
        \item Получившееся дерево может служить заменой остовного дерева в соответствующих методах кластеризации
    \end{itemize}
\end{itemize}