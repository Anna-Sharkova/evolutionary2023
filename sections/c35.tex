\section{Поддержание разнообразия популяции.}


► Пример использования: обычная оптимизация, но нужно больше разнообразия (или постановка задачи мультимодальная)
► Решение: целенаправленно понижать приспособленность особей, слишком похожих друг на друга
► Что такое «слишком похожи»?
► Похожи по приспособленности — легко реализовать, наивный подход ► Похожи по генотипам — вторая по простоте, не так наивно
► Похожи по фенотипам/поведению — в идеале надо делать так


\textbf{Базовые подходы}
► «Fitness sharing»
► Параметры: минимальное допустимое расстояние σ, функция расстояния d(i, j)
►«Sharing function»:s(i,j)=1− d(i,j) α если d(i,j)<σ,иначе 0 σ
► Пересчет приспособленности: F ′(i) = ΣF(i)βj s(i,j)
► «Crowding»: приспособленность не меняется, меняется оператор отбора 
► Restricted Tournament Selection
► Добавляем особи в новую популяцию по одной
► Для каждой особи ищем турнирным отбором похожую особь
► Если найденная похожая особь хуже, заменяем еетекущей добавляемой особью
► Deterministic Crowding
► Потомок заменяет наиболее похожего на себя родителя, если приспособленность потомка лучше

\textbf{Специализированные методы}
► В дифференциальной эволюции
► DE уже достаточно неплохо справляется с разделением на ниши
► Дополнительное улучшение: выбор ближайшего соседа (а не случайной особи)
в качестве базового решения
► Кластеризация: заимствование методов машинного обучения + модификация
операторов отбора
► Nearest Better: эвристика для кластеризации, заточенная под эволюционные вычисления
► Для каждой особи, кроме наилучшей, находим ближайшую особь, приспособленность которой лучше
► Получившееся дерево может служить заменой остовного дерева в соответствующих методах кластеризации