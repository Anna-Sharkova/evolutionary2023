\section{Стабильность постановок задач. Примеры успешного применения эволюционных алгоритмов.}

- (День 1) Давайте оптимизировать маршрут курьера от склада к клиентам: задача коммивояжера\\
- (День 2) ... только курьеров будет пятеро, мы решили расширяться\\
- (День 3) ... ой, а мы ещё на юге города склад открыли\\
- (День 4) . . . да, знаете, а еще Дима не может раньше десяти выехать, а у Тамары по пятницам в семь вечера тренинг\\
- ... а иногда в восемь... а иногда в девять утра\\
- (День 5) ... а можно учитывать, что сорок килограммов девочки не унесут? \\
- (День 6) ... учтите, что с объемной сумкой надо медленнее идти!!\\
- (День 7) ... а давайте помнить, кто на сколько опоздал, и учитывать это?\\
- ... как учитывать? Не знаю как, давайте пока как-нибудь!\\

Когда задача коммивояжера доходит до того момента, что решать классическим алгоритмом уже не является возможным, то можно применять эволюционный алгоритм. 
Увеличение сложности = уменьшение константности 

Примеры:\\
1. NASA: Максимизация направленности антенны для околоземной орбиты\\
2. Группа Оганова, программный комплекс USPEX: Поиск стабильных соединений при высоких давлениях (примеры: Na3Cl, NaCl7, Na3Cl2)\\
3. Параметры околоземной орбиты для группировки из четырех спутников, покрывающей (почти) весь земной шар \\
    - Требует в несколько раз меньше топлива для поддержания стабильности\\
    - Использует гравитационные аномалии и возмущения, а не борется с ними\\
4. Genetic Improvement Programming: ускорение программы для выравнивания чтений на референсный геном в 77 раз по сравнению с оригиналом (на геноме человека)\\
