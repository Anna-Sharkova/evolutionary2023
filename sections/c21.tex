\section{Требования к задаче, решаемой с помощью ЭА. Общие идеи о представлениях решений.}
Для решения задачи необходимо:
\begin{itemize}
	\item знать область определения функции
	\item знать область определения значений
\end{itemize}
Это необходимо для:
\begin{itemize}
	\item Генератора случайных особей
	\item Операторов, принимающие особи и порождающие особи (мутации и скрещивание)
	\item Операторов отбора
\end{itemize}
Особенности кодировки решений:
\begin{itemize}
	\item Должно происходить кодирование всех решений (возможно кроме плохих)
	\item Одно решение может быть закодированно по разному (желательно чтобы хорошие решение имели больше кодировок чем плохие)
	\item Свойство гладкости. Малые изменения на уровне генотипа - малые изменения на уровне фенотипа
	\item Понимать что делаешь. Это необходимо для эффективной реализации кроссовера так как необходимо брать лучшие гены от родителей. Кроме того ребенок должен быть в области решения и не желательно хуже чем родители. 
\end{itemize}